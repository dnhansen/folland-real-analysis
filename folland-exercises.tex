% Document setup
\documentclass[article, a4paper, 11pt, oneside]{memoir}
\usepackage[utf8]{inputenc}
\usepackage[T1]{fontenc}
\usepackage[UKenglish]{babel}

% Document info
\newcommand\doctitle{Folland: \emph{Real Analysis}}
\newcommand\docauthor{Danny Nygård Hansen}

% Formatting and layout
\usepackage[autostyle]{csquotes}
\usepackage[final]{microtype}
\usepackage{xcolor}
\frenchspacing
\usepackage{latex-sty/articlepagestyle}
\usepackage{latex-sty/articlesectionstyle}

% Fonts
\usepackage[largesmallcaps]{kpfonts}
\DeclareSymbolFontAlphabet{\mathrm}{operators} % https://tex.stackexchange.com/questions/40874/kpfonts-siunitx-and-math-alphabets
\linespread{1.06}
\let\mathfrak\undefined
\usepackage{eufrak}
\usepackage{inconsolata}
\usepackage{amssymb}

% Hyperlinks
\usepackage{hyperref}
\definecolor{linkcolor}{HTML}{4f4fa3}
\hypersetup{%
	pdftitle=\doctitle,
	pdfauthor=\docauthor,
	colorlinks,
	linkcolor=linkcolor,
	citecolor=linkcolor,
	urlcolor=linkcolor,
	bookmarksnumbered=true
}

% Equation numbering
\numberwithin{equation}{chapter}

% Footnotes
\footmarkstyle{\textsuperscript{#1}\hspace{0.25em}}

% Mathematics
\usepackage{latex-sty/basicmathcommands}
\usepackage{latex-sty/framedtheorems}
\usepackage{latex-sty/topologycommands}
\usepackage{tikz-cd}
\usetikzlibrary{babel}

% Lists
\usepackage{enumitem}
\setenumerate[0]{label=\normalfont(\alph*)}

% Bibliography
\usepackage[backend=biber, style=authoryear, maxcitenames=2, useprefix]{biblatex}
\addbibresource{references.bib}

% Title
\title{\doctitle}
\author{\docauthor}

\newcommand{\setF}{\mathbb{F}}
\newcommand{\ev}{\mathrm{ev}}
\newcommand{\calT}{\mathcal{T}}
\newcommand{\calU}{\mathcal{U}}
\newcommand{\calB}{\mathcal{B}}
\newcommand{\calE}{\mathcal{E}}
\newcommand{\calC}{\mathcal{C}}
\newcommand{\calD}{\mathcal{D}}
\newcommand{\calF}{\mathcal{F}}
\newcommand{\calG}{\mathcal{G}}
\newcommand{\calM}{\mathcal{M}}
\newcommand{\calA}{\mathcal{A}}
\newcommand{\calP}{\mathcal{P}}
\newcommand{\calR}{\mathcal{R}}
\newcommand{\borel}{\mathcal{B}}
\newcommand{\measurable}{\mathcal{M}}
\newcommand{\wto}{\Rightarrow}
\DeclarePairedDelimiter{\net}{\langle}{\rangle}
\newcommand{\strucS}{\mathfrak{S}}
\DeclarePairedDelimiter{\gen}{\langle}{\rangle} % Generating set
\newcommand{\frakL}{\mathfrak{L}}


%% Framed exercise environment

\mdfdefinestyle{swannexercise}{%
    skipabove=0.5em plus 0.4em minus 0.2em,
	skipbelow=0.5em plus 0.4em minus 0.2em,
	leftmargin=-5pt,
	rightmargin=-5pt,
	innerleftmargin=5pt,
	innerrightmargin=5pt,
	innertopmargin=5pt,
	innerbottommargin=4pt,
	linewidth=0pt,
	splittopskip=1.2em minus 0.2em,
	splitbottomskip=0.5em plus 0.2em minus 0.1em,
	backgroundcolor=backgroundcolor,
	frametitlebackgroundcolor=titlecolor,
	frametitlefont={\scshape},
    theoremseparator={\space\thechapter},
    theoremspace={.},
	frametitleaboveskip=3pt,
	frametitlebelowskip=2pt
}

\mdtheorem[style=swannexercise]{exerciseframed}{Exercise}

\let\oldexerciseframed\exerciseframed
\renewcommand{\exerciseframed}{%
  \crefalias{theorem}{exerciseframed}%
  \oldexerciseframed}

\makeatother

\newcommand{\solutionlabelfont}[1]{{\normalfont\color{linkcolor}#1}}
\newlist{solution}{enumerate}{1}
\setlist[solution]{leftmargin=0pt, parsep=0pt, listparindent=\parindent, font=\solutionlabelfont, label=(\alph*), labelsep=0pt, labelwidth=20pt, itemindent=20pt, align=left, itemsep=10pt}


\begin{document}

\maketitle

\chapter{Measures}

\addtocounter{section}{1}
\section[sigma-algebras][$\sigma$-algebras]{$\sigma$-algebras}

\begin{exerciseframed*}[1]
    Let $\calM$ be an infinite $\sigma$-algebra.
    %
    \begin{enumerate}
        \item $\calM$ contains an infinite sequence of disjoint sets.
        \item $\card{\calM} \geq \mathfrak{c}$.
    \end{enumerate}
\end{exerciseframed*}

\begin{solution}
	\item We show by contraposition that there exists an $A \in \calM$ such that the restriction of $\calM$ to $A^c$ is infinite. That is, assuming that no such set exists, we show that $\calM$ is finite. Pick any nonempty $A \in \calM$. Then the restriction of $\calM$ to $A$ and $A^c$ respectively are both finite. For any $B \in \calM$ we can write
	%
	\begin{equation*}
		B = (B \intersect A) \union (B \intersect A^c).
	\end{equation*}
	%
	But each set in the union lies in one of the restrictions, so there are finitely many decompositions like the one above, so there are finitely many sets $B \in \calM$.
	
	Now construct the sequence: Pick $A \in \calM$ as above, restrict $\calM$ to $A^c$, and continue recursively.
	
	\item Let $(A_n)$ be the sequence constructed above. There is an injection $\phi\colon 2^\naturals \to \calM$ given by $\phi(I) = \bigunion_{i \in I} A_i$ (injectivity follows since the sets in the sequence are disjoint). Hence $\card{\calM} \geq \card{\powerset\naturals} = \mathfrak{c}$.
\end{solution}


% \subsection{Exercise 5}



\section{Measures}

\begin{exerciseframed*}[14]
    If $\mu$ is a semifinite measure and $\mu(E) = \infty$, for any $C > 0$ there exists $F \subseteq E$ with $C < \mu(F) < \infty$.
\end{exerciseframed*}
%
Consider
%
\begin{equation*}
	S = \sup \set{ \mu(F) }{ F \subseteq E, \mu(F) < \infty }.
\end{equation*}
%
If $S = \infty$, then the result is obvious. So assume towards a contradiction that $S < \infty$. For $n \in \naturals$ choose $F_n \subseteq E$ with $\mu(F_n) < \infty$ such that
%
\begin{equation*}
	S - \frac{1}{n} \leq \mu(F_n) \leq S.
\end{equation*}
%
Put $G_k = \bigunion_{n=1}^k F_n$. Then $G_k \subseteq E$ and $\mu(G_k) < \infty$, so the same inequality holds with $F_n$ replaced by $G_k$. Now putting $G = \bigunion_{k \in \naturals} G_k$, continuity of $\mu$ gives
%
\begin{equation*}
	S - \frac{1}{n} \leq \mu(G) \leq S
\end{equation*}
%
for all $n \in \naturals$, so $\mu(G) = S$.

By assumption $\mu(E \setminus G) = \infty$, so $E \setminus G$ contains a set $G' \in \calM$ such that $0 < \mu(G') < \infty$. But then
%
\begin{equation*}
	\mu(G \union G') = \mu(G) + \mu(G') > S,
\end{equation*}
%
a contradiction.



\begin{exerciseframed*}[16]
    Let $(X,\calM,\mu)$ be a measure space. A set $E \subseteq X$ is called \emph{locally measurable} if $E \intersect A \in \calM$ for all $A \in \calM$ such that $\mu(A) < \infty$. Let $\widetilde{\calM}$ be the collection of all locally measurable sets. Clearly $\calM \subseteq \widetilde{\calM}$; if $\calM = \widetilde{\calM}$, then $\mu$ is called \emph{saturated}.
    %
    \begin{enumerate}[label=(\alph*)]
        \item If $\mu$ is $\sigma$-finite, then $\mu$ is saturated.
        \item $\widetilde{\calM}$ is a $\sigma$-algebra.
        \item Define $\tilde\mu$ on $\widetilde{\calM}$ by $\tilde\mu(E) = \mu(E)$ if $E \in \calM$ and $\tilde\mu(E) = \infty$ otherwise. Then $\tilde\mu$ is a saturated measure on $\widetilde{\calM}$, called the \emph{saturation} of $\mu$.
        \item If $\mu$ is complete, so is $\tilde\mu$.
        \item Suppose that $\mu$ is semifinite. For $E \in \widetilde{\calM}$ define
        %
        \begin{equation*}
            \underline\mu(E)
                = \sup\set{\mu(A)}{\text{$A \in \calM$ and $A \subseteq E$}}.
        \end{equation*}
        %
        Then $\underline\mu$ is a saturated measure on $\widetilde{\calM}$ that extends $\mu$.

        \item Let $X_1,X_2$ be disjoint uncountable sets, $X = X_1 \union X_2$, and $\calM$ the $\sigma$-algebra of countable or co-countable sets in $X$. Let $\mu_0$ be counting measure on $\powerset{X_1}$, and define $\mu$ on $\calM$ by $\mu(E) = \mu_0(E \intersect X_1)$. Then $\mu$ is a measure on $\calM$, $\widetilde\calM = \powerset{X}$, and in the notation of parts (c) and (e), $\tilde\mu \neq \underline\mu$.
    \end{enumerate}
\end{exerciseframed*}

\begin{solution}%[label=(\alph*)]
	\item Assume that $\mu$ is $\sigma$-finite, and let $E \subseteq X$ be locally measurable. Let $(A_n) \subseteq \calM$ be such that $X = \bigunion_{n \in \setN} A_n$ and $\mu(A_n) < \infty$. Then $E \intersect A_n \in \calM$, and so $E = \bigunion_{n \in \setN} (E \intersect A_n) \in \calM$.
	
	\item Clearly we have $X \in \widetilde\calM$. Then let $(E_n) \subseteq \widetilde\calM$, and let $A \in \calM$ with $\mu(A) < \infty$. Then
	%
	\begin{equation*}
		A \intersect \bigunion_{n \in \setN} E_n
			= \bigunion_{n \in \setN} (A \intersect E_n)
			\in \calM,
	\end{equation*}
	%
	so $\bigunion_{n \in \setN} E_n \in \widetilde\calM$. Finally let $E \in \widetilde\calM$ and $A \in \calM$ with $\mu(A) < \infty$. Then
	%
	\begin{equation*}
		E^c \intersect A
			= (E \intersect A)^c \intersect A
			\in \calM
	\end{equation*}
	%
	since $E \intersect A \in \calM$, so $E^c \in \widetilde\calM$.

	\item We first show that $\tilde\mu$ is a measure. Clearly $\tilde\mu(\emptyset) = 0$, so let $(E_n)$ be a sequence of disjoint sets in $\widetilde\calM$, and let $E = \bigunion_{n\in\naturals} E_n$. Say that $E_m$ does not lie in $\calM$ for some $m \in \naturals$. Then we must have $\tilde\mu(E) = \infty$, since otherwise $E \in \calM$ with $\mu(E) < \infty$, and hence $E_m = E_m \intersect E \in \calM$. Thus we have
    %
    \begin{equation*}
        \sum_{n=1}^\infty \tilde\mu(E_n)
            \geq \tilde\mu(E_m)
            = \infty
            = \tilde\mu(E),
    \end{equation*}
    %
    so $\sum_{n=1}^\infty \tilde\mu(E_n) = \tilde\mu(E)$. The same is obviously true if all $E_n$ lie in $\calM$.
    
    Next we show that $\tilde\mu$ is saturated, i.e. that $\widetilde{\widetilde\calM} \subseteq \widetilde\calM$, so let $E \in \widetilde{\widetilde\calM}$. For all $A \in \widetilde{\calM}$ with $\tilde\mu(A) < \infty$ we then have $E \intersect A \in \widetilde{\calM}$. By definition of $\tilde\mu$ we must have $A \in \calM$, so we also have
	%
	\begin{equation*}
		E \intersect A
			= (E \intersect A) \intersect A
			\in \calM.
	\end{equation*}
	%
	And since this is true for all $A \in \calM$ with $\mu(A) < \infty$, it follows that $E \in \widetilde{\calM}$.

	\item Assume that $\mu$ is complete. Let $F \subseteq X$ be such that there is a set $E \in \widetilde{\calM}$ with $F \subseteq E$ and $\tilde\mu(E) = 0$. Then also $E \in \calM$, and since $\mu$ is complete we have $F \in \calM \subseteq \widetilde{\calM}$ as desired.

	\item Assume that $\mu$ is semifinite. We first show that $\underline\mu$ is a measure. Clearly $\underline\mu(\emptyset) = 0$, so let $(E_n) \subseteq \widetilde{\calM}$ be a sequence of disjoint sets. Clearly $\underline\mu$ is increasing, so sigma-additivity is obvious if any of the sets $E_n$ have infinite measure. Assume then that $\underline\mu(E_n) < \infty$ for all $n \in \setN$. Let $\epsilon > 0$, and choose $A_n \in \calM$ such that $A_n \subseteq E_n$ and $\underline\mu(E_n) \leq \mu(A_n) + \epsilon/2^n$. Then
	%
	\begin{equation*}
		\underline\mu \bigl( \bigunion_{n \in \setN} E_n \bigr)
			\geq \mu \bigl( \bigunion_{n \in \setN} A_n \bigr)
			= \sum_{n=1}^\infty \mu(A_n)
			\geq \sum_{n=1}^\infty \mu(E_n) - \epsilon.
	\end{equation*}
	%
	Since this holds for all $\epsilon > 0$, we obtain the first inequality. For the other inequality, let $E = \bigunion_{n \in \setN} E_n$, and first assume that $\underline\mu(E) = \infty$. Pick $A \in \calM$ with $A \subseteq E$. Since $\mu$ is semifinite, we can choose $A$ such that $C < \mu(A) < \infty$ for any given $C > 0$. Letting $A_n = A \intersect E_n \in \calM$ we get
	%
	\begin{equation*}
		C
			< \mu(A)
			= \sum_{n=1}^\infty \mu(A_n)
			\leq \sum_{n=1}^\infty \underline\mu(E_n),
	\end{equation*}
	%
	and since $C$ is arbitrary, we get $\sum_{n=1}^\infty \underline\mu(E_n) = \infty$. If instead $\underline\mu(E) < \infty$, pick $A \subseteq E$ with $A \in \calM$ and $\underline\mu(E) \leq \mu(A) + \epsilon$. Again letting $A_n = A \intersect E_n$ we get
	%
	\begin{equation*}
		\underline\mu(E) - \epsilon
			\leq \mu(A)
			= \sum_{n=1}^\infty \mu(A_n)
			\leq \sum_{n=1}^\infty \underline\mu(E_n).
	\end{equation*}
	%
	And since $\epsilon$ is arbitrary, we obtain the other inequality.

	Next we show that $\underline\mu$ is saturated. Letting $E$ be locally $\underline\mu$-measurable, we must show that $E$ is also locally $\mu$-measurable. So let $A \in \calM$ with $\mu(A) < \infty$. Then $\underline\mu(A) < \infty$, and so $E \intersect A \in \widetilde{\calM}$. But then
	%
	\begin{equation*}
		E \intersect A
			= (E \intersect A) \intersect A
			\in \calM,
	\end{equation*}
	%
	as desired.

	\item It is pretty obvious that $\mu$ is a measure on $\calM$. Then let $E \subseteq X$ and $A \in \calM$ with $\mu(A) < \infty$. Then $A \intersect X_1$ must be finite, and so $A$ is not co-countable. But then it is countable, and so is $E \intersect A$, hence $E \intersect A \in \calM$. Thus every subset of $X$ is locally measurable.
	
	Notice that $\mu$ is semifinite. We have $\tilde\mu(X_2) = \infty$ since $X_2 \not\in \calM$, but $\underline\mu(X_2) = 0$ since every subset of $X_2$ is disjoint from $X_1$, and so is has measure zero.
\end{solution}


\section{Outer Measures}

\subsection{Exercise 17}

The inequality $\leq$ holds by definition. For the other inequality, notice that
%
\begin{align*}
	\mu^* \bigl( E \intersect \bigunion_{j \in \setN} A_j \bigr)
		&= \mu^*(E \intersect A_1) + \mu^* \bigl( E \intersect \bigunion_{j=2}^\infty A_j \bigr) \\
		&= \sum_{j=1}^n \mu^*(E \intersect A_j) + \mu^* \bigl( E \intersect \bigunion_{j=n+1}^\infty A_j \bigr) \\
		&\geq \sum_{j=1}^n \mu^*(E \intersect A_j)
\end{align*}
%
for all $n \in \setN$. Letting $n \to \infty$ proves the inequality.


\begin{exerciseframed*}[18]
    Let $\calA \subseteq \powerset{X}$ be an algebra, $\calA_\sigma$ the collection of countable unions of sets in $\calA$, and $\calA_{\sigma\delta}$ the collection of countable intersections of sets in $\calA_\sigma$. Let $\mu_0$ be a premeasure on $\calA$ and $\mu^*$ the induced outer measure.
    %
    \begin{enumerate}
        \item For any $E \subseteq X$ and $\epsilon > 0$ there exists $A \in \calA_\sigma$ with $E \subseteq A$ with $\mu^*(A) \leq \mu^*(E) + \epsilon$.

        \item If $\mu^*(E) < \infty$, then $E$ is $\mu^*$-measurable iff there exists $B \in \calA_{\sigma\delta}$ with $E \subseteq B$ and $\mu^*(B \setminus E) = 0$.

        \item If $\mu_0$ is $\sigma$-finite, the restriction $\mu^*(E) < \infty$ in (b) is superfluous.
    \end{enumerate}
\end{exerciseframed*}


\begin{solution}
	\item Let $E \subseteq X$ and $\epsilon > 0$. The definition of $\mu^*$ yields a sequence $(A_n) \subseteq \calA$ such that $E \subseteq \bigunion_{n\in\naturals} A_n$ and $\sum_{n=1}^\infty \mu_0(A_n) \leq \mu^*(E) + \epsilon$. It follows that
	%
	\begin{equation*}
		\mu^*(E) + \epsilon
			\geq \sum_{n=1}^\infty \mu_0(A_n)
			= \sum_{n=1}^\infty \mu^*(A_n)
			\geq \mu^* \Bigl( \bigunion_{n \in \setN} A_n \Bigr).
	\end{equation*}

	\item Let $E \subseteq X$. For $n \in \setN$ there is a set $B_n \in \calA_\sigma$ such that $E \subseteq B_n$ and $\mu^*(B_n) \leq \mu^*(E) + 1/n$. Letting $B = \bigintersect_{n \in \setN} B_n \in \calA_{\sigma\delta}$ we get $\mu^*(B) \leq \mu^*(E)$, and since $E \subseteq B$ we also have the opposite inequality, so $\mu^*(B) = \mu^*(E)$.
	
	Now assume that $\mu^*(E) < \infty$ and that $E$ is $\mu^*$-measurable. Then
	%
	\begin{equation*}
		\mu^*(B)
			= \mu^*(B \intersect E) + \mu^*(B \intersect E^c)
			= \mu^*(E) + \mu^*(B \setminus E),
	\end{equation*}
	%
	from which it follows that $\mu^*(B \setminus E) = 0$.

	Conversely, assume that there is a $B \in \calA_{\sigma\delta}$ with $E \subseteq B$ and $\mu^*(B \setminus E) = 0$. Then $B$ lies in the $\sigma$-algebra generated by $\calA$, so it is $\mu^*$-measurable. Let $A \subseteq X$. Then
	%
	\begin{align*}
		\mu^*(A \intersect E^c)
			&\leq \mu^*(B \intersect A \intersect E^c) + \mu^*(B^c \intersect A \intersect E^c) \\
			&= \mu^*(A \intersect (B \union E)^c) \\
			&= \mu^*(A \intersect B^c),
	\end{align*}
	%
	and so
	%
	\begin{equation*}
		\mu^*(A \intersect E) + \mu^*(A \intersect E^c)
			\leq \mu^*(A \intersect B) + \mu^*(A \intersect B^c)
			= \mu^*(A),
	\end{equation*}
	%
	showing that $E$ is $\mu^*$-measurable. (Notice that we haven't used that $\mu^*(E) < \infty$ for the second implication.)

	\item We only need to prove the first implication above. By $\sigma$-finiteness of $\mu_0$, let $(E_n)$ be a sequence of subsets of $X$ such that $\mu^*(E_n) < \infty$ and $E = \bigunion_{n \in \setN} E_n$. Let $\epsilon > 0$. Then there are sets $A_n \in \calA_\sigma$ such that $\mu^*(A_n) \leq \mu^*(E_n) + \epsilon/2^n$. Letting $B_\epsilon = \bigunion_{n \in \setN} A_n \in \calA_\sigma$ we get
	%
	\begin{equation*}
		\mu^*(B_\epsilon \setminus E)
			= \mu^* \Bigl( \bigunion_{n \in \setN} (A_n \intersect E^c) \Bigr)
			\leq \mu^* \Bigl( \bigunion_{n \in \setN} (A_n \intersect E_n^c) \Bigr)
			\leq \sum_{n=1}^\infty \mu^*(A_n \setminus E_n)
			\leq \epsilon.
	\end{equation*}
	%
	Finally we let $B = \bigintersect_{k \in \setN} B_{1/k} \in \calA_{\sigma\delta}$, and we get $\mu^*(B \setminus E) = 0$ as desired.
\end{solution}

\begin{remark*}
    Notice that (b) and (c) in particular show that any Lebesgue measurable set $E$, and therefore any Borel set, is the union of a $G_\delta$ set $B$ and a Lebesgue null set $B \setminus E$.
\end{remark*}


\begin{exerciseframed*}[20]
    Let $\mu^*$ be an outer measure on $X$, $\calM^*$ the $\sigma$-algebra of $\mu^*$-measurable sets, $\overline\mu = \mu^*|_{\calM^*}$, and $\mu^+$ the outer measure induced by $\overline\mu$ as in (1.12) (with $\overline\mu$ and $\calM^*$ replacing $\mu_0$ and $\calA$).
    %
    \begin{enumerate}
        \item If $E \subseteq X$, we have $\mu^*(E) \leq \mu^+(E)$, with equality iff there exists $A \in \calM^*$ with $A \supseteq E$ and $\mu^*(A) = \mu^*(E)$.

        \item If $\mu^*$ is induced from a premeasure, then $\mu^* = \mu^+$.

        \item If $X = \{0,1\}$, there exists an outer measure $\mu^*$ on $X$ such that $\mu^* \neq \mu^+$.
    \end{enumerate}
\end{exerciseframed*}

\begin{solution}
    \item Recall that the definition of $\mu^+$ means that
    %
    \begin{equation*}
        \mu^+(E)
            = \inf \set[\bigg]{ \sum_{n=1}^\infty \overline\mu(A_n) }{ A_n \in \calM^*, E \subseteq \bigunion_{n\in\naturals} A_n },
    \end{equation*}
    %
    and that we by definition of $\overline\mu$ can replace $\overline\mu$ with $\mu^*$. For any such sequence $(A_n)$ we have
    %
    \begin{equation*}
        \mu^*(E)
            \leq \mu^* \Bigl( \bigunion_{n\in\naturals} A_n \Bigr)
            \leq \sum_{n=1}^\infty \mu^*(A_n)
            = \sum_{n=1}^\infty \overline\mu(A_n).
    \end{equation*}
    %
    And since $\mu^+(E)$ is the infimum of all such sums, we have $\mu^*(E) \leq \mu^+(E)$.

    Next assume that there is an $A \in \calM^*$ with $E \subseteq A$ such that $\mu^*(A) = \mu^*(E)$. Using the sequence $A_1 = A$ and $A_n = \emptyset$ for $n > 1$ in the definition of $\mu^+$ yields
    %
    \begin{equation*}
        \mu^+(E)
            \leq \overline\mu(A)
            = \mu^*(A)
            = \mu^*(E).
    \end{equation*}
    %
    Hence $\mu^+(E) = \mu^*(E)$ as desired.

    Conversely, assuming that $\mu^*(E) = \mu^+(E)$ we have
    %
    \begin{equation*}
        \mu^*(E)
            = \inf \set[\bigg]{ \sum_{n=1}^\infty \mu^*(A_n) }{ A_n \in \calM^*, E \subseteq \bigunion_{n\in\naturals} A_n }.
    \end{equation*}
    %
    Given $\epsilon > 0$, choose a sequence $(A_n)$ such that
    %
    \begin{equation*}
        \mu^* \Bigl( \bigunion_{n\in\naturals} A_n \Bigr)
            \leq \sum_{n=1}^\infty \mu^*(A_n)
            \leq \mu^*(E) + \epsilon,
    \end{equation*}
    %
    and let $B_\epsilon = \bigunion_{n\in\naturals} A_n$. Letting $A = \bigintersect_{k\in\naturals} B_{1/k} \in \calM^*$ we thus have $\mu^*(A) \leq \mu^*(E)$.

    \item Assume that $\mu^*$ is induced from a premeasure on an algebra $\calA$, and let $E \subseteq X$. Recall that $\calA$ consists of $\mu^*$-measurable sets, so $\sigma(\calA) \subseteq \calM^*$. For $n \in \naturals$ choose, in accordance with Exercise~1.18(a), a set $A_n \in \calA_\sigma$ with $E \subseteq A_n$ such that $\mu^*(A_n) \leq \mu^*(E) + 1/n$. Letting $A = \bigintersect_{n\in\naturals} A_n$ we have $E \subseteq A$ and $\mu^*(A) \leq \mu^*(E)$. The other inequality is obvious, so $\mu^*(A) = \mu^*(E)$, and part (a) implies that $\mu^*(E) = \mu^+(E)$ as desired.
\end{solution}


\begin{exerciseframed*}[21]
    Let $\mu^*$ be an outer measure induced from a premeasure and $\overline\mu$ the restriction of $\mu^*$ to the $\mu^*$-measurable sets. Then $\overline\mu$ is saturated.
\end{exerciseframed*}
%
Let $\calA$ denote the algebra on which the premeasure in question is defined, and denote by $\calM^*$ the $\sigma$-algebra of $\mu^*$-measurable sets. Recall that $\calA \subseteq \calM^*$.

Let $E \subseteq X$ be locally measurable. It suffices to show that
%
\begin{equation*}
    \mu^*(F)
        \geq \mu^*(F \intersect E) + \mu^*(F \intersect E^c)
\end{equation*}
%
for all $F \subseteq X$ with $\mu^*(F) < \infty$. Given $\epsilon   > 0$, Exercise~1.18(a) yields a set $A \in \calA_\sigma$ such that $\mu^*(A) \leq \mu^*(F) + \epsilon$. Then $\mu^*(A) < \infty$, and so $E \intersect A \in \calM^*$. It follows that
%
\begin{align*}
    \mu^*(F) + \epsilon
        \geq \mu^*(A)
        &= \mu^*(A \intersect (E \intersect A))
          + \mu^*(A \intersect (E \intersect A)^c) \\
        &= \mu^*(A \intersect E) + \mu^*(A \intersect E^c) \\
        &\geq \mu^*(F \intersect E) + \mu^*(F \intersect E^c),
\end{align*}
%
and hence $E \in \calM^*$. Thus $\overline\mu$ is saturated.


\begin{exerciseframed*}[22]
    Let $(X,\calM,\mu)$ be a measure space, $\mu^*$ the outer measure induced b $\mu$ according to (1.12), $\calM^*$ the $\sigma$-algebra of $\mu^*$-measurable sets, and $\overline\mu = \mu^*|_{\calM^*}$.
    %
    \begin{enumerate}
        \item If $\mu$ is $\sigma$-finite, then $\overline\mu$ is the completion of $\mu$.
        \item In general, $\overline\mu$ is the saturation of the completion of $\mu$.
    \end{enumerate}
\end{exerciseframed*}

\begin{solution}
    \item Let $\overline\calM$ be the $\sigma$-algebra from Theorem~1.9 (namely, the $\sigma$-algebra generated by the sets in $\calM$ along with all $\mu$-null sets). This is clearly the smallest $\sigma$-algebra on which there can exist a complete extension of $\mu$, so since $\overline\mu$ is also a complete extension of $\mu$, we must have $\overline\calM \subseteq \calM^*$. Theorem~1.9 yields the uniqueness of a complete extension of $\mu$ on $\overline\calM$, so it suffices to show that $\calM^* \subseteq \overline\calM$.
    
    Now assume that $\mu$ is $\sigma$-finite, and let $E \in \calM^*$. Then also $E^c \in \calM^*$, and Exercise~1.18(c) ensures the existence of sets $B,D \in \calM_{\sigma\delta} = \calM$ with $E \subseteq B$ and $E^c \subseteq D$ such that
    %
    \begin{equation*}
        \mu^*(B \setminus E) = 0
        \quad \text{and} \quad
        \mu^*(E \setminus D^c) = \mu^*(D \setminus E^c) = 0.
    \end{equation*}
    %
    It follows that
    %
    \begin{equation*}
        \mu(B \setminus D^c)
            \leq \mu^*(B \setminus E) + \mu^*(E \setminus D^c)
            = 0,
    \end{equation*}
    %
    so $E \setminus D^c$ is a $\mu$-null set. Thus $E = D^c \union (E \setminus D^c)$ is a union of a set in $\calM$ and a $\mu$-null set, and hence $E \in \overline\calM$.

    \item Let $\hat\mu$ denote the completion of $\mu$ on $\overline\calM$, and let $\widetilde\calM$ denote the $\sigma$-algebra of locally $\hat\mu$-measurable sets. First we show that $\widetilde\calM = \calM^*$, so let $E \in \widetilde\calM$. To show that $E$ is $\mu^*$-measurable it suffices to show that
    %
    \begin{equation*}
        \mu^*(F)
            \geq \mu^*(F \intersect E) + \mu^*(F \intersect E^c)
    \end{equation*}
    %
    for all $F \subseteq X$ with $\mu^*(F) < \infty$. Calculations identical to the ones in the solution to Exercise~1.21 show this.

    Conversely, let $E \in \calM^*$ and consider $A \in \overline\calM$ with $\hat\mu(A) < \infty$. Then also $A \in \calM^*$, so $E \intersect A \in \calM^*$. The argument at the beginning of part (a) showed that $\overline\mu$ is an extension of $\hat\mu$, so $\mu^*(E \intersect A) = \hat\mu(E \intersect A) < \infty$. The same argument as in part (a), only now using Exercise~1.18(b) instead of (c), shows that $E \intersect A \in \overline\calM$, and so $E \in \widetilde\calM$.

    Finally, let $\tilde\mu$ denote the saturation of $\hat\mu$. We show that $\overline\mu = \tilde\mu$. Since the completion of $\mu$ on $\overline\calM$ is unique, the two measures must agree here. Instead let $E \in \widetilde\calM \setminus \overline\calM$. By definition of $\tilde\mu$ we must then have $\tilde\mu(E) = \infty$. On the other hand, we just showed (for $E \intersect A$ instead of $E$) that $\mu^*(E) < \infty$ implies $E \in \overline\calM$. Since we have assumed that this is not the case, we must have $\overline\mu(E) = \mu^*(E) = \infty$. Thus $\overline\mu = \tilde\mu$.
\end{solution}


\begin{exerciseframed*}[25]
    If $E \subseteq \reals$, the following are equivalent.
    %
    \begin{enumerate}
        \item $E \in \calM_\mu$.
        \item $E = V \setminus N_1$ where $V$ is a $G_\delta$ set and $\mu(N_1) = 0$.
        \item $E = H \union N_2$ where $H$ is an $F_\sigma$ set and $\mu(N_2) = 0$.
    \end{enumerate}
\end{exerciseframed*}
%
Folland proves this claim when $\mu(E) < \infty$, so assume that $\mu(E) = \infty$. Since $\mu$ is $\sigma$-finite, there is a sequence $(E_n)_{n\in\naturals}$ in $\calM_\mu$ with $\mu(E_n) < \infty$ for all $n \in \naturals$ and $E = \bigunion_{n\in\naturals} E_n$. Then there are sequences $(H_n)$ of $F_\sigma$ sets and $(N_n)$ of null sets such that $E_n = H_n \union N_n$. Then $H = \bigunion_{n\in\naturals} H_n$ is also an $F_\sigma$ set and $N = \bigunion_{n\in\naturals} N_n$ a null set, and $E = H \union N$.

Applying this to $E^c$ yields a similar decomposition $E^c = H \union N$. But then $E = H^c \setminus N$, and $H^c$ is a $G_\delta$ set.


\chapter{Integration}

\begin{exerciseframed*}[10]
    The following implications are valid iff the measure $\mu$ is complete:
    %
    \begin{enumerate}
        \item If $f$ is measurable and $f = g$ $\mu$-a.e., then $g$ is measurable.
        \item If $f_n$ is measurable for $n \in \naturals$ and $f_n \to f$ $\mu$-a.e., then $f$ is measurable.
    \end{enumerate}
\end{exerciseframed*}

\begin{solution}
    \item Let $f,g \colon (X,\calE,\mu) \to (Y,\calF)$ be functions from a measure space to a measurable space where $f$ is $(\calE,\calF)$-measurable. Let $N = \{f \neq g\}$ and assume that $\mu(N) = 0$. Given $B \in \calF$ we must show that $g\preim(B) \in \calE$. But notice that
    %
    \begin{equation*}
        g\preim(B)
            = f\preim(B) \union \{ f \not\in B, g \in B \} \setminus \{ f \in B, g \not\in B \},
    \end{equation*}
    %
    and that the latter two sets are subsets of $N$, hence measurable. Thus $g\preim(B)$ is also measurable.

    Conversely, let $\mu$ be a measure on a measurable space $(X,\calE)$ that is not complete, and let $N \subseteq X$ be a non-measurable $\mu$-null set. Then $\indicator{N} = 0$ $\mu$-a.e., but $\indicator{N}$ is not measurable.

    \item Consider the set $A$ of points $x \in X$ such that $f_n(x)$ does not converge to $f(x)$. Then $f_n \indicator{A^c} \to f \indicator{A^c}$ pointwise everywhere, so Proposition~2.7 (or Corollary~2.9) implies that $f \indicator{A^c}$ is measurable. By assumption $\mu(A) = 0$, so $f \indicator{A^c} = f$ $\mu$-a.e., so part (a) implies that $f$ is measurable.
    
    Conversely
\end{solution}


\chapter{Signed Measures and Differentiation}

\section{Signed Measures}

\subsection{Exercise 2}

Assume that $E$ is $\nu$-null, and let $P \union N$ be a Hahn decomposition for $\nu$. Then
%
\begin{equation*}
	\nu^+(E) = \nu(E \intersect P) = 0,
\end{equation*}
%
since $E \intersect P \subseteq E$. Similarly we get $\nu^-(E) = 0$, so $\abs{\nu}(E) = 0$.

Conversely, assume that $\abs{\nu}(E) = 0$. Then $\nu^\pm(F) = 0$ for all measurable $F \subseteq E$, and so $\nu(F) = 0$.

The other claims follow directly from the above.


\subsection{Exercise 3}

\begin{enumerate}[label=(\alph*)]
	\item This follows directly from the definition of $L^1(\nu)$.
	
	\item This is a simple application of the triangle inequality.

	\item If $\abs{f} \leq 1$, then
	%
	\begin{equation*}
		\abs[\bigg]{ \int_E f \dif\nu }
			\leq \int_E \abs{f} \dif\abs{\nu}
			\leq \abs{\nu}(E),
	\end{equation*}
	%
	showing one inequality. For the other inequality, let $P \union N$ be a Hahn decomposition of $\nu$, and let $f = \indicator{P} - \indicator{N}$. Then $\int_E f \dif\nu = \abs{\nu}(E)$.
\end{enumerate}


\subsection{Exercise 4}

Let $P \union N$ be a Hahn decomposition for $\nu$. Then
%
\begin{equation*}
	\nu^+(E)
		= \nu(E \intersect P)
		\leq \lambda(E \intersect P)
		\leq \lambda(E),
\end{equation*}
%
and similarly for $\nu^-$.


\subsection{Exercise 5}

First notice that
%
\begin{equation*}
	\nu_1 + \nu_2
		= (\nu_1^+ + \nu_2^+) - (\nu_1^- + \nu_2^-),
\end{equation*}
%
so by the previous exercise we have
%
\begin{equation*}
	\abs{\nu_1 + \nu_2}
		= (\nu_1 + \nu_2)^+ + (\nu_1 + \nu_2)^+
		\leq (\nu_1^+ + \nu_2^+) + (\nu_1^- + \nu_2^-)
		= \abs{\nu_1} + \abs{\nu_2}.
\end{equation*}

\end{document}